% Spring 基础篇
% 
% zzy.hit@gmail.com
% 2013/05/24

\documentclass[xcolor=dvipsnames]{beamer}
\usetheme{Warsaw}
\setbeamertemplate{items}[circle]
\definecolor{links}{HTML}{2A1B81}
\hypersetup{colorlinks,linkcolor=,urlcolor=links}
\usepackage{CJKutf8}
\usepackage{verbatim}

\title{Introduction to Spring}
\subtitle{Inversion of Control and Aspect Oriented Programming}
\author{zzy.hit@gmail.com}
\date{\today}

\begin{document}
\begin{CJK}{UTF8}{gbsn}


  % title page 
  \frame{\titlepage}

  % outline
  \frame{\frametitle{Outline}
    \begin{itemize}
      \item IoC - Inversion of Control
      \item Spring简介
      \item Spring入门
      \item AOP - Aspect-Oriented Programming
    \end{itemize}
  }

  % IoC
  \frame{
    \begin{center}
      IoC - Inversion of Control
    \end{center}
  }

  \frame{\frametitle{IoC (Inversion of Control)}
    \begin{theorem}
      高层不应依赖低层,模块都必须抽象
    \end{theorem}
    \verbatiminput{ioc1.c}
    \pause
    \textcolor{red}{now, save to database?}
    \pause
    \begin{itemize}
    \item 高层逻辑通常是抽象业务,应该具有可重用性
    \item 不应依赖底层某个模块实现
    \end{itemize}
  }

  \frame{\frametitle{IoC (Inversion of Control)}
    \begin{theorem}
      实现必须依赖抽象,而不是抽象依赖实现
    \end{theorem}
    \verbatiminput{ioc2_1.java}
    \pause
    \textcolor{red}{now, save to database?}
  }

  \frame{\frametitle{IoC (Inversion of Control)}
    \verbatiminput{ioc2_2.java}
    \pause
    \textcolor{red}{FileWriter? DatabaseWriter? NetWriter? CompositeWriter...}
  }

  \frame{\frametitle{IoC (Inversion of Control)}
    
    \begin{theorem}
      应用不应依赖容器,容器应服务于应用
    \end{theorem}

    \begin{itemize}
    \item 业务逻辑不做修改可迁移
    \item 容器不侵入应用
    \end{itemize}

    \begin{center}
      \includegraphics[width=0.3\textwidth]{spring_logo_mini.png}
      \includegraphics[width=0.3\textwidth]{pico_container_logo.png}

      \includegraphics[width=0.3\textwidth]{seasar_logo_blue.png}
      \includegraphics[width=0.3\textwidth]{guice_logo.png}
    \end{center}
  }

  % uncle bob
  \frame{\frametitle{Questions?}
    \begin{center}
      \includegraphics[width=0.8\textwidth]{uncle_bob.jpg}

      \href{http://www.objectmentor.com/resources/articles/dip.pdf}{The Dependency Inversion Principle}

    \end{center}
  }

  % Sample
  \frame{\frametitle{举个例子}
    \begin{center}
      \includegraphics[width=0.8\textwidth]{sample.jpg}
    \end{center}
  }

  % spring
  \frame{
    \begin{center}
      \includegraphics[width=0.5\textwidth]{spring_logo.png}
    \end{center}
  }

  \frame{\frametitle{Spring简介}
    Spring的核心是一个轻量级的IoC (Inversion of Control) 容器。用它构建的应用可以达到模块间低耦合,可测试,最终使得整个系统结构简化,易于维护。

    \begin{itemize}
    \item 轻量级(Lightweight)

      体积小(核心不足1M)、资源少、无侵入(Nonintrusive)
    \item 容器(Container)

      管理组件生命周期、状态、依赖关系等
    \item IoC (Inversion of Control)

      通过配置维护组件依赖关系,无需编码
    \end{itemize}
  }

  \frame{\frametitle{Spring简介}
    除此之外,Spring的目标是实现一个全方位的整合框架,其下由多个子框架组合,子框架之间彼此独立,并可以使用其他框架方案代替。

    \begin{itemize}
    \item AOP框架

      Spring神器之一(Aspect-Oriented Programming)
    \item 持久层

      如JDBC、ORM(Hibernate、iBatis)、事务处理等
    \item Web框架

      Spring提供了自己的Web框架,同时你也可以用其他框架来代替,如Struts、Webwork等
    \end{itemize}
  }

  \frame{\frametitle{Spring Framework}
    \begin{center}
      \includegraphics[width=0.9\textwidth]{spring_framework.png}
    \end{center}
  }

  \frame{\frametitle{工作方式}
    \begin{center}
      \includegraphics[width=0.8\textwidth]{spring_container.png}
    \end{center}
  }

  % Pack app using Spring
  \frame{\frametitle{配置 - Bean定义}
    \begin{itemize}
      \item id/name, class,
      \item constructor-arg, properties, value/ref/value-ref
      \item abstract/parent
    \end{itemize}
  }

  % factory/builder
  \frame{\frametitle{BeanFactory}
    XmlBeanFactory
  }

  % 
  \frame{\frametitle{LifeCycle}
    \begin{itemize}
      \item init-method - InitializingBean
      \item destroy-method - DisposableBean
      \item lazy-init
    \end{itemize}
  }

  \frame{\frametitle{FactoryBean}
    factory-bean, factory-method
  }

  % (util:list/map/properties)

  \frame{\frametitle{ApplicationContext}

    \begin{center}
      \includegraphics[width=1.0\textwidth]{beanfactory.jpg}
    \end{center}
  }

  \frame{\frametitle{Scope}
    scope(sigleton/prototype/request/session/global),
  }

  \frame{\frametitle{Annotation}
    \begin{itemize}
      \item what is annotation
      \item @Service, @Resource(Autowired)
    \end{itemize}
  }

  \frame{\frametitle{Configuration (xml vs annotation)}

  }

  % AOP
  \frame{
    \begin{center}
      AOP - Aspect Oriented Programming
    \end{center}
  }

  % Aspect (Big Spring is watching you)
  % proxy

  \frame{\frametitle{Homework}
      %CodeBean, CodeListLoader#load()#getCodeBeans()
      DBCodeBeanListLoader vs MappedCodeBeanListLoader
  }

  \frame{\frametitle{推荐资料}

    \begin{itemize}

      \item \href{http://www.jdon.com/designpatterns/}{Design Pattern}
      \item \href{http://www.tutorialspoint.com/spring/index.htm}{Spring Tutorial}
      \item \href{http://book.douban.com/subject/1426848/}{Expert One-on-One J2EE Development without EJB}
        \includegraphics[width=0.3\textwidth]{expert_one_on_one.jpg}
    \end{itemize}
  }


  \bgroup
  \usenavigationsymbolstemplate{}
  \setbeamercolor{background canvas}{bg=black}
  \frame[plain]{
    \begin{center}
      \includegraphics[width=0.8\textwidth]{continue.jpg}
    \end{center}
  }
  \egroup

\end{CJK}
\end{document}
