% Mybatis 基础篇
% 
% zzy.hit@gmail.com
% 2013/05/24

\documentclass[xcolor=dvipsnames]{beamer}
\usetheme{Warsaw}
\setbeamertemplate{items}[circle]
\definecolor{links}{HTML}{2A1B81}
\hypersetup{colorlinks,linkcolor=,urlcolor=links}
\usepackage{CJKutf8}
\usepackage{verbatim}

\title{Introduction to Mybatis}
\author{zzy.hit@gmail.com}
\date{\today}

\begin{document}
\begin{CJK}{UTF8}{gbsn}


  % title page 
  \frame{\titlepage}

  \frame{\frametitle{概要}
    \begin{itemize}
    \item Mybatis简介
    \item 基础使用
      \begin{itemize}
      \item SQL配置文件
      \item 动态SQL
      \item 接口方式
      \end{itemize}
    \item 扩展使用
      \begin{itemize}
      \item TypeHandler
      \item Plugins
      \end{itemize}
    \end{itemize}
  }

  \frame{\frametitle{目标}
    \begin{itemize}
    \item 能够编写SQL配置文件
    \item 能够使用动态SQL
    \item 能够编写TypeHandler
    \item 了解ResultHandler, Interceptor, Executor
    \end{itemize}
  }


  \bgroup
  \usenavigationsymbolstemplate{}
  \frame[plain]{
    \begin{center}
      (那是什么)
    \end{center}
  }
  \egroup

  \frame{\frametitle{起源}
    \begin{itemize}
    \item ORM (Object Relational Mapping)
    \item iBatis → 2010/06/16 → MyBatis
    \item http://code.google.com/p/mybatis
    \end{itemize}
  }

  \bgroup
  \usenavigationsymbolstemplate{}
  \frame[plain]{
    \begin{center}
      (生而为何)
    \end{center}
  }
  \egroup

  \frame{\frametitle{从JDBC说起}
    \begin{itemize}
    \item java.sql.Driver
    \item java.sql.Connection
    \item java.sql.Statement
    \item java.sql.ResultSet
    \item java.sql.SavePoint
    \end{itemize}
  }

  \frame{\frametitle{读取数据七步}
    \begin{itemize}
%      \pause
    \item 参数检查
%      \pause
    \item 获取数据连接
%      \pause
    \item 查询准备
%      \pause
    \item 绑定参数
%      \pause
    \item 执行查询
%      \pause
    \item 绑定结果
%      \pause
    \item 释放资源
    \end{itemize}
  }

  \frame[plain]{
    \begin{center}
      (using JDBC)
    \end{center}
  }
    
  \frame{\frametitle{观察vs 抽象}
    \begin{itemize}
    \item 数据归类
    \item 行为归接口
    \item 披着接口的抽象类实现
    \item Sigleton? (Never do this!!)
    \end{itemize}
  }

  \frame{\frametitle{观察变化点}
    \verbatiminput{../pattern1.code}
  }

  \frame{\frametitle{为什么痛苦}
    \begin{itemize}
    \item import java.sql.SQLException
    \item if (id!=null) { where.append(id=?) }
    \item DAO with business logic.
    \end{itemize}
  }

  \frame{\frametitle{我们真正需要什么}
    \begin{itemize}
    \item SQL是一种可重用的逻辑资源
    \item Query
      \begin{itemize}
        \item 参数
        \item 结果(单个/多个)
      \end{itemize}
    \item Update
      \begin{itemize}
      \item 参数
      \end{itemize}
    \end{itemize}
  }

  \frame{
    \verbatiminput{../pattern2.code}
  }

  \frame[plain]{
    \begin{block}{Google}
      肯定有人遇到过同样的问题,并已经解决
    \end{block}
  }

  \frame[plain]{
    \begin{center}
      (using MyBatis)
    \end{center}
  }

  \frame{\frametitle{MyBatis}
    \begin{itemize}
    \item SQL配置文件(id为资源标识)
    \item SqlSession(统一访问行为)
    \item Dynamic SQL(重用逻辑片段)
    \end{itemize}
  }

  \frame{\frametitle{自增主键}
    \verbatiminput{../auto_increment.code}
  }

  \frame{\frametitle{Dynamic SQL}
    \verbatiminput{../dynamic_sql.code}
  }

  \frame{\frametitle{foreach}
    \verbatiminput{../foreach.code}
  }

  \frame{\frametitle{limit vs offset}
    \verbatiminput{../limit_offset.code}
  }

  \frame{\frametitle{typeHandler}
    \verbatiminput{../typehandler.code}
  }

  \frame{\frametitle{一些例子}
    \begin{itemize}
      \item IntegerTypeHandler
      \item StringTypeHandler
      \item EnumTypeHandler
      \item ArrayTypeHandler
    \end{itemize}
  }

  \frame{\frametitle{ResultHandler}
    \verbatiminput{../result_handler.code}
  }

  \frame{\frametitle{Homework}
    \verbatiminput{../homework.code}
  }

  \frame[plain]{
    \begin{center}
      -EOF-
    \end{center}
  }
\end{CJK}
\end{document}
